\section{Experiments}

The following experiments refer to \emph{linearly} and \emph{nonlinearly} separable generated datasets of size 100.

The Python source code is available at: \href{https://github.com/dmeoli/optiml}{\texttt{github.com/dmeoli/optiml}}.

\subsection{Support Vector Classifier}

Below experiments are about the SVC for which I tested different values for the regularization hyperparameter $C$, i.e., from \emph{soft} to \emph{hard margin}, and in case of nonlinearly separable data also different \emph{kernel functions} mentioned above.

The experiments about SVCs are available at: \\ \href{https://github.com/dmeoli/optiml/blob/master/notebooks/optimization/CM_SVC_report_experiments.ipynb}{\texttt{github.com/dmeoli/optiml/blob/master/notebooks/optimization/CM\_SVC\_report\_experiments.ipynb}}.

\subsubsection{Hinge loss}

\paragraph{Primal formulation}

The experiments results shown in~\ref{primal_l1_svc_cv_results} referred to \emph{Stochastic Gradient Descent} algorithm are obtained with $\alpha$, i.e., the \emph{learning rate} or \emph{step size}, setted to 0.001 and $\beta$, i.e., the \emph{momentum}, equal to 0.4. The batch size is setted to 20. Training is stopped if after 5 iterations the training loss is not lower than the best found so far.

\begin{table}[H]
\centering
\caption{Primal $\protect \mathcal{L}_1$-SVC results}
\label{primal_l1_svc_cv_results}
\begin{tabular}{lllrrrr}
\toprule
          &   &     &  fit\_time &  accuracy &  n\_iter &  n\_sv \\
solver & momentum & C &           &           &         &       \\
\midrule
sgd & none & 1   &  0.472888 &     0.970 &     313 &    53 \\
          &   & 10  &  0.479242 &     0.985 &     383 &    19 \\
          &   & 100 &  0.259484 &     0.980 &     206 &    10 \\
          & standard & 1   &  0.314042 &     0.970 &     228 &    48 \\
          &   & 10  &  0.370457 &     0.985 &     295 &    16 \\
          &   & 100 &  0.151304 &     0.980 &     123 &    11 \\
          & nesterov & 1   &  0.292495 &     0.970 &     228 &    48 \\
          &   & 10  &  0.320098 &     0.985 &     287 &    16 \\
          &   & 100 &  0.161079 &     0.985 &     129 &    11 \\
liblinear & - & 1   &  0.001720 &     0.985 &     332 &    15 \\
          &   & 10  &  0.001161 &     0.985 &     554 &     5 \\
          &   & 100 &  0.003379 &     0.985 &    1000 &     7 \\
\bottomrule
\end{tabular}
\end{table}


The results provided from the \emph{custom} implementation, i.e., the SGD with different momentum settings, are strongly similar to those of \emph{sklearn} implementation, i.e., \emph{liblinear}~\cite{fan2008liblinear} implementation, in terms of \emph{accuracy} score. More training data points are selected as \emph{support vectors} from the SGD solver but it always requires lower iterations, i.e., epochs, to achieve the same \emph{numerical precision}. \emph{Standard} or \emph{Polyak} and \emph{Nesterov} momentums always perform lower iterations as expected from the theoretical analysis of the convergence rate.

\begin{figure}[H]
	\centering
	\includegraphics[scale=0.55]{img/l1_svc_loss_history}
	\caption{SGD Convergence for the Primal formulation of the $\protect \mathcal{L}_1$-SVC}
	\label{fig:l1_svc_history}
\end{figure}

\paragraph{Linear Dual formulations}

The experiments results shown in~\ref{linear_lagrangian_dual_l1_svc_cv_results} are obtained with $\alpha$, i.e., the \emph{learning rate} or \emph{step size}, setted to 0.001 for the \emph{AdaGrad} algorithm. Notice that the \emph{unreg\_bias} dual refers to the formulation~\eqref{eq:svc_lagrangian_dual}, while the \emph{reg\_bias} dual refers to the formulation~\eqref{eq:svc_bcqp_lagrangian_dual}.

\begin{table}[H]
\centering
\caption{Wolfe Dual linear $\protect \mathcal{L}_1$-SVC results}
\label{linear_dual_l1_svc_cv_results}
\begin{tabular}{llrrrr}
\toprule
       &      &  fit\_time &  accuracy &  n\_iter &  n\_sv \\
solver & C &           &           &         &       \\
\midrule
smo & 0.1  &  0.068648 &     0.985 &      33 &    38 \\
       & 1.0  &  0.065627 &     0.980 &      62 &    17 \\
       & 10.0 &  0.119814 &     0.980 &     295 &    10 \\
libsvm & 0.1  &  0.002770 &     0.985 &      37 &    38 \\
       & 1.0  &  0.002411 &     0.985 &     243 &    17 \\
       & 10.0 &  0.002456 &     0.985 &     194 &    10 \\
cvxopt & 0.1  &  0.022841 &     0.985 &       9 &    38 \\
       & 1.0  &  0.038121 &     0.980 &      10 &    17 \\
       & 10.0 &  0.039981 &     0.980 &      10 &    11 \\
\bottomrule
\end{tabular}
\end{table}


For what about the linear \emph{Wolfe dual} formulation we can immediately notice as higher \emph{regularization hyperparameter} $C$ makes the model harder, so the \emph{custom} implementation of the SMO algorithm and also the \emph{sklearn} implementation, i.e., \emph{libsvm}~\cite{chang2011libsvm} implementation, needs to perform more iterations to achieve the same \emph{numerical precision}; meanwhile the \emph{cvxopt}~\cite{vandenberghe2010cvxopt} seems to be insensitive to the increasing complexity of the model. The results in terms of \emph{accuracy} and number of \emph{support vectors} are strongly similar to each others.

\begin{table}[H]
\centering
\caption{Lagrangian Dual linear $\protect \mathcal{L}_1$-SVC results}
\label{linear_lagrangian_dual_l1_svc_cv_results}
\begin{tabular}{llrrrr}
\toprule
           &     &  fit\_time &  accuracy &  n\_iter &  n\_sv \\
dual & C &           &           &         &       \\
\midrule
reg\_bias & 1   &  0.019050 &     0.985 &       1 &   194 \\
           & 10  &  0.007226 &     0.985 &       1 &   194 \\
           & 100 &  0.015732 &     0.985 &       1 &   194 \\
unreg\_bias & 1   &  0.007376 &     0.985 &       1 &   195 \\
           & 10  &  0.008295 &     0.985 &       1 &   195 \\
           & 100 &  0.017933 &     0.985 &       1 &   195 \\
\bottomrule
\end{tabular}
\end{table}


For what about the linear \emph{Lagrangian dual} formulation we can see as it seems to be insensitive to the increasing complexity of the model in terms of number of \emph{iterations} but it tends to select many training data points as \emph{support vectors}.

\paragraph{Nonlinear Dual formulations}

The experiments results shown in~\ref{nonlinear_dual_l1_svc_cv_results} and~\ref{nonlinear_lagrangian_dual_l1_svc_cv_results} are obtained with \emph{d} and \emph{r} hyperparameters equal to 3 and 1 respectively for the \emph{polynomial} kernel; \emph{gamma} is setted to \emph{`scale`} for both \emph{polynomial} and \emph{gaussian RBF} kernels. The experiments results shown in~\ref{nonlinear_lagrangian_dual_l1_svc_cv_results} are obtained with $\alpha$, i.e., the \emph{learning rate} or \emph{step size}, setted to 0.001 for the \emph{AdaGrad} algorithm.

\begin{table}[H]
\centering
\caption{Wolfe Dual nonlinear $\protect \mathcal{L}_1$-SVC results}
\label{nonlinear_dual_l1_svc_cv_results}
\begin{tabular}{lllrrrr}
\toprule
       &     &     &  fit\_time &  accuracy &  n\_iter &  n\_sv \\
solver & kernel & C &           &           &         &       \\
\midrule
smo & poly & 1   &  0.394293 &    0.6825 &     143 &    30 \\
       &     & 10  &  0.301946 &    0.9475 &      65 &    10 \\
       &     & 100 &  0.216001 &    0.9775 &      38 &     6 \\
       & rbf & 1   &  0.360697 &    1.0000 &      66 &    51 \\
       &     & 10  &  0.252868 &    1.0000 &      38 &    13 \\
       &     & 100 &  0.275172 &    1.0000 &      56 &    12 \\
libsvm & poly & 1   &  0.004139 &    1.0000 &     233 &    30 \\
       &     & 10  &  0.003570 &    1.0000 &     118 &    10 \\
       &     & 100 &  0.005292 &    1.0000 &      88 &     6 \\
       & rbf & 1   &  0.004747 &    1.0000 &     252 &    50 \\
       &     & 10  &  0.003243 &    1.0000 &     134 &    13 \\
       &     & 100 &  0.002651 &    1.0000 &     182 &    12 \\
cvxopt & poly & 1   &  0.292866 &    0.6775 &      10 &    31 \\
       &     & 10  &  0.317473 &    0.9475 &      10 &    10 \\
       &     & 100 &  0.309560 &    0.9775 &      10 &     6 \\
       & rbf & 1   &  0.249617 &    1.0000 &      10 &    49 \\
       &     & 10  &  0.232780 &    1.0000 &      10 &    14 \\
       &     & 100 &  0.257583 &    1.0000 &      10 &    17 \\
\bottomrule
\end{tabular}
\end{table}


\begin{table}[H]
\centering
\caption{Lagrangian Dual nonlinear $\protect \mathcal{L}_1$-SVC results}
\label{nonlinear_lagrangian_dual_l1_svc_cv_results}
\begin{tabular}{lllrrrr}
\toprule
           &     &      &    fit\_time &  accuracy &  n\_iter &  n\_sv \\
dual & kernel & C &             &           &         &       \\
\midrule
reg\_bias & poly & 0.1  &  248.217142 &    1.0000 &   50000 &   212 \\
           &     & 1.0  &  248.144591 &    0.6800 &   50000 &    96 \\
           &     & 10.0 &  251.175852 &    0.5125 &   50000 &   102 \\
           & rbf & 0.1  &  217.612475 &    1.0000 &   43664 &   222 \\
           &     & 1.0  &  251.638453 &    1.0000 &   50000 &    56 \\
           &     & 10.0 &  260.563878 &    1.0000 &   50000 &    16 \\
unreg\_bias & poly & 0.1  &  246.308773 &    1.0000 &   50000 &   277 \\
           &     & 1.0  &  247.420974 &    0.7675 &   50000 &   152 \\
           &     & 10.0 &  250.390498 &    0.6875 &   50000 &   144 \\
           & rbf & 0.1  &  245.312625 &    1.0000 &   50000 &   222 \\
           &     & 1.0  &  252.125890 &    1.0000 &   50000 &    51 \\
           &     & 10.0 &  259.193395 &    1.0000 &   50000 &    28 \\
\bottomrule
\end{tabular}
\end{table}


The same considerations made for the previous linear \emph{Wolfe dual} and \emph{Lagrangian dual} formulations are confirmed also in the nonlinearly separable case. In this setting the complexity of the model coming with higher $C$ regularization values seems to be not paying a tradeoff in terms of the number of \emph{iterations} of the algorithm and, moreover, the \emph{reg\_bias Lagrangian dual} formulation seems to perform better wrt the \emph{unreg\_bias} formulation, both tends to select even more training data points as \emph{support vectors}.

\begin{figure}[H]
	\centering
	\includegraphics[scale=0.55]{img/lagrangian_dual_l1_svc_loss_history}
	\caption{AdaGrad convergence for the Lagrangian Dual formulation of the Nonlinear $\protect \mathcal{L}_1$-SVC}
	\label{fig:lagrangian_dual_l1_svc_loss_history}
\end{figure}

\pagebreak

\subsubsection{Squared Hinge loss}

\paragraph{Primal formulation}

The experiments results shown in~\ref{primal_l2_svc_cv_results} referred to \emph{Stochastic Gradient Descent} algorithm are obtained with $\alpha$, i.e., the \emph{learning rate} or \emph{step size}, setted to 0.001 and $\beta$, i.e., the \emph{momentum}, equal to 0.4. The batch size is setted to 20. Training is stopped if after 5 iterations the training loss is not lower than the best found so far.

\begin{table}[H]
\centering
\caption{Primal $\protect \mathcal{L}_2$-SVC results}
\label{primal_l2_svc_cv_results}
\begin{tabular}{lllrrrr}
\toprule
          &   &      &  fit\_time &  accuracy &  n\_iter &  n\_sv \\
solver & momentum & C &           &           &         &       \\
\midrule
sgd & none & 0.1  &  0.173259 &     0.975 &     488 &    49 \\
          &   & 1.0  &  0.196758 &     0.980 &     433 &    24 \\
          &   & 10.0 &  0.041189 &     0.985 &      57 &    21 \\
          & polyak & 0.1  &  0.087903 &     0.975 &     246 &    49 \\
          &   & 1.0  &  0.127343 &     0.985 &     233 &    25 \\
          &   & 10.0 &  0.018536 &     0.985 &      23 &    20 \\
          & nesterov & 0.1  &  0.097543 &     0.975 &     247 &    49 \\
          &   & 1.0  &  0.134696 &     0.985 &     234 &    23 \\
          &   & 10.0 &  0.023753 &     0.985 &      30 &    18 \\
liblinear & - & 0.1  &  0.001174 &     0.980 &      52 &    46 \\
          &   & 1.0  &  0.001623 &     0.980 &     563 &    25 \\
          &   & 10.0 &  0.002059 &     0.980 &    1000 &    19 \\
\bottomrule
\end{tabular}
\end{table}


Again, the results provided from the \emph{custom} implementation, i.e., the SGD with different momentum settings, are strongly similar to those of \emph{sklearn} implementation, i.e., \emph{liblinear}~\cite{fan2008liblinear} implementation, in terms of \emph{accuracy} score. More training data points are selected as \emph{support vectors} from the SGD solver but it always requires even lower iterations, i.e., epochs, to achieve the same \emph{numerical precision}. \emph{Standard} or \emph{Polyak} and \emph{Nesterov} momentums always perform lower iterations as expected from the theoretical analysis of the convergence rate.

\begin{figure}[H]
	\centering
	\includegraphics[scale=0.55]{img/l2_svc_loss_history}
	\caption{SGD convergence for the Primal formulation of the $\protect \mathcal{L}_2$-SVC}
	\label{fig:l2_svc_loss_history}
\end{figure}

\pagebreak

\subsection{Support Vector Regression}

Below experiments are about the SVR for which I tested different values for regularization hyperparameter $C$, i.e., from \emph{soft} to \emph{hard margin}, the $\epsilon$ penalty value and in case of nonlinearly separable data also different \emph{kernel functions} mentioned above.

The experiments about SVRs are available at: \\ \href{https://github.com/dmeoli/optiml/blob/master/notebooks/optimization/CM_SVR_report_experiments.ipynb}{\texttt{github.com/dmeoli/optiml/blob/master/notebooks/optimization/CM\_SVR\_report\_experiments.ipynb}}.

\subsubsection{Epsilon-insensitive loss}

\paragraph{Primal formulation}

The experiments results shown in~\ref{primal_l1_svr_cv_results} referred to \emph{Stochastic Gradient Descent} algorithm are obtained with $\alpha$, i.e., the \emph{learning rate} or \emph{step size}, setted to 0.001 and $\beta$, i.e., the \emph{momentum}, equal to 0.4. The batch size is setted to 20. Training is stopped if after 5 iterations the training loss is not lower than the best found so far.

\begin{table}[H]
\centering
\caption{Primal $\protect \mathcal{L}_1$-SVR results}
\label{primal_l1_svr_cv_results}
\begin{tabular}{llllrrrr}
\toprule
          &   &     &     &  fit\_time &        r2 &  n\_iter &  n\_sv \\
solver & momentum & C & epsilon &           &           &         &       \\
\midrule
sgd & none & 1   & 0.1 &  0.351371 &  0.295083 &    1000 &   100 \\
          &   &     & 0.2 &  0.393301 &  0.295083 &    1000 &   100 \\
          &   &     & 0.3 &  0.378809 &  0.295083 &    1000 &   100 \\
          &   & 10  & 0.1 &  0.271119 &  0.983476 &     787 &    99 \\
          &   &     & 0.2 &  0.298002 &  0.983513 &     814 &    99 \\
          &   &     & 0.3 &  0.299304 &  0.983515 &     729 &    97 \\
          &   & 100 & 0.1 &  0.058686 &  0.984022 &      91 &    97 \\
          &   &     & 0.2 &  0.073681 &  0.984039 &     119 &    97 \\
          &   &     & 0.3 &  0.097403 &  0.984050 &     143 &    96 \\
          & standard & 1   & 0.1 &  0.331673 &  0.424184 &    1000 &   100 \\
          &   &     & 0.2 &  0.366600 &  0.424184 &    1000 &   100 \\
          &   &     & 0.3 &  0.366812 &  0.424184 &    1000 &   100 \\
          &   & 10  & 0.1 &  0.197872 &  0.983477 &     471 &    98 \\
          &   &     & 0.2 &  0.197098 &  0.983513 &     471 &    98 \\
          &   &     & 0.3 &  0.190229 &  0.983517 &     443 &    96 \\
          &   & 100 & 0.1 &  0.032477 &  0.984031 &      51 &    97 \\
          &   &     & 0.2 &  0.044617 &  0.984040 &      65 &    96 \\
          &   &     & 0.3 &  0.051167 &  0.984050 &      86 &    96 \\
          & nesterov & 1   & 0.1 &  0.337784 &  0.424112 &    1000 &   100 \\
          &   &     & 0.2 &  0.390105 &  0.424112 &    1000 &   100 \\
          &   &     & 0.3 &  0.367784 &  0.424112 &    1000 &   100 \\
          &   & 10  & 0.1 &  0.209850 &  0.983477 &     517 &    99 \\
          &   &     & 0.2 &  0.211796 &  0.983513 &     473 &    98 \\
          &   &     & 0.3 &  0.260021 &  0.983549 &     683 &    97 \\
          &   & 100 & 0.1 &  0.042005 &  0.984030 &      64 &    97 \\
          &   &     & 0.2 &  0.050735 &  0.984040 &      71 &    96 \\
          &   &     & 0.3 &  0.078552 &  0.984049 &      85 &    96 \\
liblinear & - & 1   & 0.1 &  0.001236 &  0.954684 &      12 &    99 \\
          &   &     & 0.2 &  0.001246 &  0.954842 &      14 &   100 \\
          &   &     & 0.3 &  0.001251 &  0.955335 &       9 &    99 \\
          &   & 10  & 0.1 &  0.001360 &  0.983893 &     114 &    97 \\
          &   &     & 0.2 &  0.000944 &  0.983910 &     164 &    97 \\
          &   &     & 0.3 &  0.000958 &  0.983885 &     150 &    98 \\
          &   & 100 & 0.1 &  0.001427 &  0.984028 &    1000 &    98 \\
          &   &     & 0.2 &  0.001421 &  0.984083 &    1000 &    98 \\
          &   &     & 0.3 &  0.002570 &  0.984039 &    1000 &    97 \\
\bottomrule
\end{tabular}
\end{table}


The results provided from the \emph{custom} implementation, i.e., the SGD with different momentum settings, are strongly similar to those of \emph{sklearn} implementation, i.e., \emph{liblinear}~\cite{fan2008liblinear} implementation, in terms of \emph{r2} score, except in case of $C$ regularization hyperparameter equals to 1 for which those of SGD are lower. Moreover, the SGD solver always requires lower iterations, i.e., epochs, for higher $C$ regularization values, i.e., for $C$ equals to 10 or 100, to achieve the same \emph{numerical precision}. Again, \emph{Standard} or \emph{Polyak} and \emph{Nesterov} momentums always perform lower iterations as expected from the theoretical analysis of the convergence rate. The results in terms of \emph{support vectors} are strongly similar to each others.

\begin{figure}[H]
	\centering
	\includegraphics[scale=0.5]{img/l1_svr_loss_history}
	\caption{SGD convergence for the Primal formulation of the $\protect \mathcal{L}_1$-SVR}
	\label{fig:l1_svr_loss_history}
\end{figure}

\pagebreak

\paragraph{Linear Dual formulations}

The experiments results shown in~\ref{linear_lagrangian_dual_l1_svr_cv_results} are obtained with $\alpha$, i.e., the \emph{learning rate} or \emph{step size}, setted to 0.001 for the \emph{AdaGrad} algorithm. Notice that the \emph{unreg\_bias} dual refers to the formulation~\eqref{eq:svr_lagrangian_dual}, while the \emph{reg\_bias} dual refers to the formulation~\eqref{eq:svr_bcqp_lagrangian_dual}.

\begin{table}[H]
\centering
\caption{Wolfe Dual linear $\protect \mathcal{L}_1$-SVR results}
\label{linear_dual_l1_svr_cv_results}
\begin{tabular}{lllrrrr}
\toprule
       &     &     &  fit\_time &        r2 &  n\_iter &  n\_sv \\
solver & C & epsilon &           &           &         &       \\
\midrule
smo & 1   & 0.1 &  0.060325 &  0.964127 &      17 &    98 \\
       &     & 0.2 &  0.071465 &  0.963707 &      18 &    96 \\
       &     & 0.3 &  0.068385 &  0.963707 &      14 &    96 \\
       & 10  & 0.1 &  0.183926 &  0.977576 &      65 &   100 \\
       &     & 0.2 &  0.383894 &  0.977573 &     749 &   100 \\
       &     & 0.3 &  0.165514 &  0.977573 &      78 &    99 \\
       & 100 & 0.1 &  0.434502 &  0.977515 &     549 &   100 \\
       &     & 0.2 &  0.461811 &  0.977496 &     723 &   100 \\
       &     & 0.3 &  0.410521 &  0.977493 &     926 &    99 \\
libsvm & 1   & 0.1 &  0.002630 &  0.964103 &      81 &    98 \\
       &     & 0.2 &  0.003279 &  0.963680 &      81 &    97 \\
       &     & 0.3 &  0.002134 &  0.963684 &      78 &    96 \\
       & 10  & 0.1 &  0.002560 &  0.977559 &     226 &   100 \\
       &     & 0.2 &  0.002932 &  0.977554 &     706 &   100 \\
       &     & 0.3 &  0.002898 &  0.977564 &     181 &    99 \\
       & 100 & 0.1 &  0.003907 &  0.977481 &    1224 &   100 \\
       &     & 0.2 &  0.003300 &  0.977450 &    2126 &   100 \\
       &     & 0.3 &  0.003670 &  0.977463 &    2680 &    99 \\
cvxopt & 1   & 0.1 &  0.094744 &  0.964068 &      10 &   100 \\
       &     & 0.2 &  0.095626 &  0.964118 &      10 &   100 \\
       &     & 0.3 &  0.105664 &  0.963452 &      10 &    99 \\
       & 10  & 0.1 &  0.131441 &  0.977576 &       8 &   100 \\
       &     & 0.2 &  0.112558 &  0.977573 &       9 &   100 \\
       &     & 0.3 &  0.074020 &  0.977573 &       9 &    99 \\
       & 100 & 0.1 &  0.053100 &  0.977515 &       8 &   100 \\
       &     & 0.2 &  0.035742 &  0.977496 &       9 &   100 \\
       &     & 0.3 &  0.027918 &  0.977493 &       9 &   100 \\
\bottomrule
\end{tabular}
\end{table}


For what about the linear \emph{Wolfe dual} formulation we can immediately notice as higher \emph{regularization hyperparameter} $C$ and lower $\epsilon$ values makes the model harder, so the \emph{custom} implementation of the SMO algorithm and also the \emph{sklearn} implementation, i.e., \emph{libsvm}~\cite{chang2011libsvm} implementation, needs to perform more iterations to achieve the same \emph{numerical precision}; meanwhile, again, the \emph{cvxopt}~\cite{vandenberghe2010cvxopt} seems to be insensitive to the increasing complexity of the model. The results in terms of \emph{r2} and number of \emph{support vectors} are strongly similar to each others.

\begin{table}[H]
\centering
\caption{Lagrangian Dual linear $\protect \mathcal{L}_1$-SVR results}
\label{linear_lagrangian_dual_l1_svr_cv_results}
\begin{tabular}{lllrrrr}
\toprule
           &     &     &   fit\_time &        r2 &  n\_iter &  n\_sv \\
dual & C & epsilon &            &           &         &       \\
\midrule
reg\_bias & 1   & 0.1 &   8.364345 &  0.954686 &   10000 &   100 \\
           &     & 0.2 &   8.774625 &  0.954848 &   10000 &   100 \\
           &     & 0.3 &   8.941515 &  0.955432 &   10000 &   100 \\
           & 10  & 0.1 &   8.744542 &  0.983899 &   10000 &    99 \\
           &     & 0.2 &  12.178584 &  0.983899 &   10000 &   100 \\
           &     & 0.3 &   8.893864 &  0.983891 &   10000 &   100 \\
           & 100 & 0.1 &   8.689983 &  0.984057 &   10000 &   100 \\
           &     & 0.2 &   8.849962 &  0.984065 &   10000 &   100 \\
           &     & 0.3 &   9.596288 &  0.984102 &   10000 &    99 \\
unreg\_bias & 1   & 0.1 &   8.299400 &  0.954391 &   10000 &   100 \\
           &     & 0.2 &   9.048550 &  0.954534 &   10000 &   100 \\
           &     & 0.3 &   8.838417 &  0.955478 &   10000 &   100 \\
           & 10  & 0.1 &  10.959342 &  0.983895 &   10000 &   100 \\
           &     & 0.2 &   9.131809 &  0.983901 &   10000 &   100 \\
           &     & 0.3 &   9.198061 &  0.983892 &   10000 &   100 \\
           & 100 & 0.1 &   8.215255 &  0.984038 &   10000 &   100 \\
           &     & 0.2 &   8.221480 &  0.984091 &   10000 &   100 \\
           &     & 0.3 &   8.801184 &  0.984093 &   10000 &   100 \\
\bottomrule
\end{tabular}
\end{table}


For what about the linear \emph{Lagrangian dual} formulation we can see as it seems to be insensitive to the increasing complexity of the model in terms of number of \emph{iterations} and require many \emph{iterations} wrt the \emph{Wolfe dual} formulation.

\begin{figure}[H]
	\centering
	\includegraphics[scale=0.55]{img/linear_lagrangian_dual_l1_svr_loss_history}
	\caption{AdaGrad convergence for the Lagrangian Dual formulation of the Linear $\protect \mathcal{L}_1$-SVR}
	\label{fig:linear_lagrangian_dual_l1_svr_loss_history}
\end{figure}

\paragraph{Nonlinear Dual formulations}

The experiments results shown in~\ref{nonlinear_dual_l1_svr_cv_results} and~\ref{nonlinear_lagrangian_dual_l1_svr_cv_results} are obtained with \emph{d} and \emph{r} hyperparameters both equal to 3 for the \emph{polynomial} kernel; \emph{gamma} is setted to \emph{`scale`} for both \emph{polynomial} and \emph{gaussian RBF} kernels. The experiments results shown in~\ref{nonlinear_lagrangian_dual_l1_svc_cv_results} are obtained with $\alpha$, i.e., the \emph{learning rate} or \emph{step size}, setted to 0.001 for the \emph{AdaGrad} algorithm.

\begin{table}[H]
\centering
\caption{Wolfe Dual nonlinear $\protect \mathcal{L}_1$-SVR formulation results}
\label{nonlinear_dual_l1_svr_cv_results}
\begin{tabular}{llllrrrr}
\toprule
       &     &     &     &     fit\_time &        r2 &    n\_iter &  n\_sv \\
solver & kernel & C & epsilon &              &           &           &       \\
\midrule
smo & poly & 1   & 0.1 &    90.455519 &  0.810056 &     47694 &    36 \\
       &     &     & 0.2 &    20.459538 &  0.671256 &      8702 &     6 \\
       &     &     & 0.3 &     9.310908 &  0.302709 &      3654 &     4 \\
       &     & 10  & 0.1 &   318.547348 &  0.736098 &    256531 &    32 \\
       &     &     & 0.2 &    51.711430 &  0.923152 &     32629 &     4 \\
       &     &     & 0.3 &     4.117346 &  0.302709 &      3654 &     4 \\
       &     & 100 & 0.1 &  2076.294908 &  0.635585 &   3294613 &    33 \\
       &     &     & 0.2 &    60.687808 &  0.923152 &     32629 &     4 \\
       &     &     & 0.3 &     7.411675 &  0.302709 &      3654 &     4 \\
       & rbf & 1   & 0.1 &     0.048293 &  0.988244 &        66 &    17 \\
       &     &     & 0.2 &     0.018928 &  0.924292 &        20 &     7 \\
       &     &     & 0.3 &     0.016849 &  0.883022 &        17 &     5 \\
       &     & 10  & 0.1 &     0.450764 &  0.989739 &       389 &    18 \\
       &     &     & 0.2 &     0.015516 &  0.924995 &        25 &     6 \\
       &     &     & 0.3 &     0.012460 &  0.882816 &        11 &     5 \\
       &     & 100 & 0.1 &     3.867796 &  0.974756 &      6664 &    19 \\
       &     &     & 0.2 &     0.026167 &  0.924995 &        25 &     6 \\
       &     &     & 0.3 &     0.011646 &  0.882816 &        11 &     5 \\
libsvm & poly & 1   & 0.1 &     0.047426 &  0.981438 &    155092 &    37 \\
       &     &     & 0.2 &     0.005325 &  0.976358 &      7326 &     6 \\
       &     &     & 0.3 &     0.003903 &  0.951282 &      3969 &     4 \\
       &     & 10  & 0.1 &     0.161407 &  0.981769 &    578347 &    32 \\
       &     &     & 0.2 &     0.020051 &  0.979414 &     28452 &     4 \\
       &     &     & 0.3 &     0.003979 &  0.951282 &      3969 &     4 \\
       &     & 100 & 0.1 &     2.261907 &  0.981844 &  13306191 &    35 \\
       &     &     & 0.2 &     0.013208 &  0.979414 &     28452 &     4 \\
       &     &     & 0.3 &     0.003728 &  0.951282 &      3969 &     4 \\
       & rbf & 1   & 0.1 &     0.004174 &  0.990088 &        96 &    17 \\
       &     &     & 0.2 &     0.023456 &  0.977763 &        36 &     7 \\
       &     &     & 0.3 &     0.001297 &  0.945601 &        24 &     5 \\
       &     & 10  & 0.1 &     0.002483 &  0.990493 &       616 &    18 \\
       &     &     & 0.2 &     0.001489 &  0.980673 &        39 &     6 \\
       &     &     & 0.3 &     0.010427 &  0.945601 &        24 &     5 \\
       &     & 100 & 0.1 &     0.005138 &  0.990496 &      9854 &    18 \\
       &     &     & 0.2 &     0.001730 &  0.980673 &        39 &     6 \\
       &     &     & 0.3 &     0.003135 &  0.945601 &        24 &     5 \\
cvxopt & poly & 1   & 0.1 &     0.102581 &  0.828482 &        10 &    37 \\
       &     &     & 0.2 &     0.111636 &  0.666571 &        10 &     6 \\
       &     &     & 0.3 &     0.084176 &  0.350876 &         9 &     4 \\
       &     & 10  & 0.1 &     0.111183 &  0.629433 &        10 &    33 \\
       &     &     & 0.2 &     0.079567 &  0.928477 &        10 &     4 \\
       &     &     & 0.3 &     0.124415 &  0.350873 &        10 &     4 \\
       &     & 100 & 0.1 &     0.124310 &  0.712681 &        10 &    36 \\
       &     &     & 0.2 &     0.129660 &  0.928478 &        10 &     4 \\
       &     &     & 0.3 &     0.141848 &  0.350876 &        10 &     4 \\
       & rbf & 1   & 0.1 &     0.043756 &  0.988117 &        10 &    17 \\
       &     &     & 0.2 &     0.033950 &  0.924679 &        10 &     7 \\
       &     &     & 0.3 &     0.032085 &  0.883386 &        10 &     5 \\
       &     & 10  & 0.1 &     0.029397 &  0.989956 &        10 &    18 \\
       &     &     & 0.2 &     0.026725 &  0.925595 &        10 &     6 \\
       &     &     & 0.3 &     0.038238 &  0.883386 &        10 &     5 \\
       &     & 100 & 0.1 &     0.020217 &  0.990216 &        10 &    40 \\
       &     &     & 0.2 &     0.032069 &  0.925595 &        10 &     6 \\
       &     &     & 0.3 &     0.028090 &  0.883386 &        10 &     5 \\
\bottomrule
\end{tabular}
\end{table}


\begin{table}[H]
\centering
\caption{Lagrangian Dual nonlinear $\protect \mathcal{L}_1$-SVR results}
\label{nonlinear_lagrangian_dual_l1_svr_cv_results}
\begin{tabular}{llllrrrr}
\toprule
           &     &     &     &     fit\_time &        r2 &   n\_iter &  n\_sv \\
dual & kernel & C & epsilon &              &           &          &       \\
\midrule
reg\_bias & poly & 1   & 0.1 &  3799.855584 &  0.759735 &  4345854 &    45 \\
           &     &     & 0.2 &  4101.699981 & -0.593191 &  3979752 &    14 \\
           &     &     & 0.3 &  3736.716269 &  0.950884 &  3667706 &    12 \\
           &     & 10  & 0.1 &  3227.341450 &  0.889292 &  4273682 &    43 \\
           &     &     & 0.2 &  3066.053793 &  0.932512 &  3833638 &    17 \\
           &     &     & 0.3 &  2973.237179 &  0.405079 &  3666349 &    14 \\
           &     & 100 & 0.1 &  5389.059721 & -0.454737 &  7000000 &    43 \\
           &     &     & 0.2 &  3120.464042 &  0.932512 &  3833638 &    17 \\
           &     &     & 0.3 &  5343.459601 &  0.405079 &  3666349 &    14 \\
           & rbf & 1   & 0.1 &    47.461531 &  0.988828 &    53076 &    19 \\
           &     &     & 0.2 &    19.048212 &  0.924431 &    17252 &     7 \\
           &     &     & 0.3 &    10.817777 &  0.894204 &    13637 &     6 \\
           &     & 10  & 0.1 &   517.461881 &  0.989716 &   636053 &    18 \\
           &     &     & 0.2 &    14.603766 &  0.916999 &    17749 &     7 \\
           &     &     & 0.3 &    12.477623 &  0.921671 &    12840 &    10 \\
           &     & 100 & 0.1 &  4853.336714 &  0.980079 &  5920433 &    18 \\
           &     &     & 0.2 &    13.425201 &  0.916999 &    17749 &     7 \\
           &     &     & 0.3 &    10.280788 &  0.921671 &    12840 &    10 \\
unreg\_bias & poly & 1   & 0.1 &  3861.455070 &  0.878100 &  4300698 &    43 \\
           &     &     & 0.2 &  3443.375626 & -1.808568 &  3718479 &    15 \\
           &     &     & 0.3 &  2860.392630 &  0.944686 &  3569608 &    12 \\
           &     & 10  & 0.1 &  3349.425094 &  0.845000 &  4372132 &    45 \\
           &     &     & 0.2 &  3226.271276 &  0.849851 &  3940737 &    15 \\
           &     &     & 0.3 &  3054.851898 &  0.931490 &  3624333 &    13 \\
           &     & 100 & 0.1 &  5480.802903 & -0.461495 &  7000000 &    43 \\
           &     &     & 0.2 &  3184.266593 &  0.849851 &  3940737 &    15 \\
           &     &     & 0.3 &  3287.755984 &  0.931490 &  3624333 &    13 \\
           & rbf & 1   & 0.1 &    65.452607 &  0.987761 &    74401 &    20 \\
           &     &     & 0.2 &    35.227437 &  0.903375 &    31217 &     9 \\
           &     &     & 0.3 &    19.617716 &  0.911973 &    25336 &     7 \\
           &     & 10  & 0.1 &   482.568146 &  0.988548 &   598918 &    19 \\
           &     &     & 0.2 &    24.508186 &  0.929011 &    31397 &     7 \\
           &     &     & 0.3 &    20.715078 &  0.911975 &    24386 &     7 \\
           &     & 100 & 0.1 &  4407.457835 &  0.886070 &  5388026 &    19 \\
           &     &     & 0.2 &    24.151294 &  0.929011 &    31397 &     7 \\
           &     &     & 0.3 &    19.131887 &  0.911975 &    24386 &     7 \\
\bottomrule
\end{tabular}
\end{table}


The same considerations made for the previous linear \emph{Wolfe dual} and \emph{Lagrangian dual} formulations are confirmed also in the nonlinearly separable case. In this setting, the complexity of the model coming with higher $C$ regularization hyperparameters and lower $\epsilon$ values pays a larger tradeoff in terms of the number of \emph{iterations} of the algorithm.

\begin{figure}[H]
	\centering
	\includegraphics[scale=0.55]{img/poly_lagrangian_dual_l1_svr_loss_history}
	\caption{AdaGrad convergence for the Lagrangian Dual formulation of the Polynomial $\protect \mathcal{L}_1$-SVR}
	\label{fig:poly_lagrangian_dual_l1_svr_loss_history}
\end{figure}

\begin{figure}[H]
	\centering
	\includegraphics[scale=0.55]{img/gaussian_lagrangian_dual_l1_svr_loss_history}
	\caption{AdaGrad convergence for the Lagrangian Dual formulation of the Gaussian $\protect \mathcal{L}_1$-SVR}
	\label{fig:gaussian_lagrangian_dual_l1_svr_loss_history}
\end{figure}

\pagebreak

\subsubsection{Squared Epsilon-insensitive loss}

\paragraph{Primal formulation}

The experiments results shown in~\ref{primal_l2_svr_cv_results} referred to \emph{Stochastic Gradient Descent} algorithm are obtained with $\alpha$, i.e., the \emph{learning rate} or \emph{step size}, setted to 0.001 and $\beta$, i.e., the \emph{momentum}, equal to 0.4. The batch size is setted to 20. Training is stopped if after 5 iterations the training loss is not lower than the best found so far.

\begin{table}[H]
\centering
\caption{Primal $\protect \mathcal{L}_2$-SVR results}
\label{primal_l2_svr_cv_results}
\begin{tabular}{llllrrrr}
\toprule
          &   &     &     &  fit\_time &        r2 &  n\_iter &  n\_sv \\
solver & momentum & C & epsilon &           &           &         &       \\
\midrule
sgd & none & 1   & 0.1 &  2.527874 &  0.984109 &    3283 &   100 \\
          &   &     & 0.2 &  2.260561 &  0.984109 &    3294 &   100 \\
          &   &     & 0.3 &  1.846209 &  0.984109 &    3321 &    98 \\
          &   & 10  & 0.1 &  0.280950 &  0.984133 &     409 &    98 \\
          &   &     & 0.2 &  0.303813 &  0.984133 &     410 &    98 \\
          &   &     & 0.3 &  0.299412 &  0.984133 &     411 &    98 \\
          &   & 100 & 0.1 &  0.060678 &  0.984133 &      47 &    98 \\
          &   &     & 0.2 &  0.064470 &  0.984133 &      47 &    98 \\
          &   &     & 0.3 &  0.073545 &  0.984133 &      47 &    98 \\
          & polyak & 1   & 0.1 &  1.144745 &  0.984109 &    2011 &   100 \\
          &   &     & 0.2 &  1.092027 &  0.984109 &    2018 &   100 \\
          &   &     & 0.3 &  1.082388 &  0.984109 &    2035 &    98 \\
          &   & 10  & 0.1 &  0.172013 &  0.984133 &     241 &    98 \\
          &   &     & 0.2 &  0.187925 &  0.984133 &     242 &    98 \\
          &   &     & 0.3 &  0.198103 &  0.984133 &     243 &    98 \\
          &   & 100 & 0.1 &  0.052779 &  0.984133 &      40 &    98 \\
          &   &     & 0.2 &  0.055219 &  0.984133 &      40 &    98 \\
          &   &     & 0.3 &  0.058916 &  0.984133 &      40 &    98 \\
          & nesterov & 1   & 0.1 &  1.112495 &  0.984109 &    2015 &   100 \\
          &   &     & 0.2 &  1.083714 &  0.984109 &    2022 &   100 \\
          &   &     & 0.3 &  1.078303 &  0.984109 &    2039 &    98 \\
          &   & 10  & 0.1 &  0.185957 &  0.984133 &     247 &    98 \\
          &   &     & 0.2 &  0.201911 &  0.984133 &     248 &    98 \\
          &   &     & 0.3 &  0.209882 &  0.984133 &     248 &    98 \\
          &   & 100 & 0.1 &  0.036464 &  0.984133 &      27 &    98 \\
          &   &     & 0.2 &  0.038866 &  0.984133 &      27 &    98 \\
          &   &     & 0.3 &  0.041946 &  0.984133 &      27 &    98 \\
liblinear & - & 1   & 0.1 &  0.001924 &  0.984109 &      84 &   100 \\
          &   &     & 0.2 &  0.001350 &  0.984109 &      84 &   100 \\
          &   &     & 0.3 &  0.002135 &  0.984109 &      84 &    98 \\
          &   & 10  & 0.1 &  0.006217 &  0.984133 &     778 &    98 \\
          &   &     & 0.2 &  0.005483 &  0.984133 &     773 &    98 \\
          &   &     & 0.3 &  0.006050 &  0.984133 &     773 &    98 \\
          &   & 100 & 0.1 &  0.048187 &  0.984133 &    7296 &    99 \\
          &   &     & 0.2 &  0.048013 &  0.984133 &    7434 &    98 \\
          &   &     & 0.3 &  0.036768 &  0.984133 &    7262 &    98 \\
\bottomrule
\end{tabular}
\end{table}


Again, the results provided from the \emph{custom} implementation, i.e., the SGD with different momentum settings, are strongly similar to those of \emph{sklearn} implementation, i.e., \emph{liblinear}~\cite{fan2008liblinear} implementation, in terms of \emph{r2} score. SGD solver always requires even lower iterations, i.e., epochs, for higher $C$ regularization values, i.e., for $C$ equals to 10 or 100, to achieve the same \emph{numerical precision}. \emph{Standard} or \emph{Polyak} and \emph{Nesterov} momentums always perform lower iterations as expected from the theoretical analysis of the convergence rate.

\begin{figure}[H]
	\centering
	\includegraphics[scale=0.5]{img/l2_svr_loss_history}
	\caption{SGD convergence for the Primal formulation of the $\protect \mathcal{L}_2$-SVR}
	\label{fig:l2_svr_loss_history}
\end{figure}
