\section{Experiments}

The following experiments refer to 3-fold cross-validation over \emph{linearly} and \emph{nonlinearly} separable generated datasets of size 100, so the reported results are to considered as a mean over the 3 folds.

\subsection{Support Vector Classifier}

Below experiments are about the SVC for which I tested different values for the regularization hyperparameter $C$, i.e., from \emph{soft} to \emph{hard margin}, and in case of nonlinearly separable data also different \emph{kernel functions} mentioned above.

\subsubsection{Hinge loss}

\paragraph{Primal formulation}

The experiments results shown in \ref{primal_svc_hinge_cv_results} referred to \emph{Stochastic Gradient Descent} algorithm are obtained with $\alpha$, i.e., the \emph{learning rate} or \emph{step size}, setted to 0.001 and $\beta$, i.e., the \emph{momentum}, equal to 0.4. The batch size is setted to 20. Training is stopped if after 5 iterations the training loss is not lower than the best found so far.

\begin{table}[H]
\centering
\caption{SVC Primal formulation results with Hinge loss}
\label{primal_svc_hinge_cv_results}
\begin{tabular}{lllrrrr}
\toprule
          &   &     &  fit\_time &  accuracy &  n\_iter &  n\_sv \\
solver & momentum & C &           &           &         &       \\
\midrule
sgd & none & 1   &  0.399575 &     0.970 &     314 &    53 \\
          &   & 10  &  0.480145 &     0.985 &     384 &    19 \\
          &   & 100 &  0.258918 &     0.980 &     207 &    10 \\
          & standard & 1   &  0.305286 &     0.970 &     229 &    48 \\
          &   & 10  &  0.350044 &     0.985 &     296 &    16 \\
          &   & 100 &  0.204298 &     0.980 &     124 &    11 \\
          & nesterov & 1   &  0.301714 &     0.970 &     229 &    48 \\
          &   & 10  &  0.345403 &     0.985 &     288 &    16 \\
          &   & 100 &  0.159904 &     0.985 &     130 &    11 \\
liblinear & - & 1   &  0.001039 &     0.985 &     332 &    15 \\
          &   & 10  &  0.001164 &     0.985 &     554 &     5 \\
          &   & 100 &  0.001599 &     0.985 &    1000 &     7 \\
\bottomrule
\end{tabular}
\end{table}


The results provided from the \emph{custom} implementation, i.e., the SGD with different momentum settings, are strongly similar to those of \emph{sklearn} implementation, i.e., \emph{liblinear} \cite{fan2008liblinear} implementation, in terms of \emph{accuracy} score. More training data points are selected as \emph{support vectors} from the SGD solver but it always requires lower iterations, i.e., epochs, to achieve the same \emph{numerical precision}. \emph{Standard} or \emph{Polyak} and \emph{Nesterov} momentums always perform lower iterations as expected from the theoretical analysis of the convergence rate.

\pagebreak

\paragraph{Linear Dual formulations}

The experiments results shown in \ref{linear_lagrangian_dual_svc_cv_results} are obtained with $\alpha$, i.e., the \emph{learning rate} or \emph{step size}, setted to 0.5 for the \emph{AdaGrad} algorithm. Notice that the \emph{qp} dual refers to the formulation \eqref{eq:svc_lagrangian_dual}, while the \emph{bcqp} dual refers to the formulation \eqref{eq:svc_bcqp_lagrangian_dual}.

\begin{table}[H]
\centering
\caption{Linear SVC Wolfe Dual formulation results with Hinge loss}
\label{linear_dual_svc_cv_results}
\begin{tabular}{llrrrr}
\toprule
       &     &  fit\_time &  accuracy &  n\_iter &  n\_sv \\
solver & C &           &           &         &       \\
\midrule
smo & 1   &  0.240170 &      0.97 &     142 &    20 \\
       & 10  &  2.656202 &      0.97 &    6975 &    16 \\
       & 100 &  2.029251 &      0.97 &    3601 &    16 \\
libsvm & 1   &  0.002645 &      0.97 &     174 &    20 \\
       & 10  &  0.002526 &      0.97 &     448 &    16 \\
       & 100 &  0.003148 &      0.97 &    2357 &    16 \\
cvxopt & 1   &  0.144664 &      0.97 &      10 &    20 \\
       & 10  &  0.162875 &      0.97 &      10 &    16 \\
       & 100 &  0.038245 &      0.97 &      10 &    18 \\
\bottomrule
\end{tabular}
\end{table}


For what about the linear \emph{Wolfe dual} formulation we can immediately notice as the \emph{regularization hyperparameter} $C$ makes the model harder, so the \emph{custom} implementation of the SMO algorithm and also the \emph{sklearn} implementation, i.e., \emph{libsvm} \cite{chang2011libsvm} implementation, needs to perform more iterations to achieve the same \emph{numerical precision}; meanwhile the \emph{cvxopt} \cite{vandenberghe2010cvxopt} seems to be insensitive to the increasing complexity of the model. The results in terms of \emph{accuracy} and number of \emph{support vectors} are strongly similar to each others.

\begin{table}[H]
\centering
\caption{Linear SVC Lagrangian Dual formulation results with Hinge loss}
\label{linear_lagrangian_dual_svc_cv_results}
\begin{tabular}{llrrrr}
\toprule
     &     &  fit\_time &  accuracy &  n\_iter &  n\_sv \\
dual & C &           &           &         &       \\
\midrule
qp & 1   &  0.028132 &      0.97 &       1 &   194 \\
     & 10  &  0.027381 &      0.97 &       1 &   194 \\
     & 100 &  0.034051 &      0.97 &       1 &   194 \\
bcqp & 1   &  0.031179 &      0.97 &       1 &   193 \\
     & 10  &  0.023645 &      0.97 &       1 &   193 \\
     & 100 &  0.028539 &      0.97 &       1 &   193 \\
\bottomrule
\end{tabular}
\end{table}


For what about the linear \emph{Lagrangian dual} formulation we can see as it seems to be insensitive to the increasing complexity of the model in terms of number of \emph{iterations} but it tends to select many training data points as \emph{support vectors}.

\pagebreak

\paragraph{Nonlinear Dual formulations}

The experiments results shown in \ref{nonlinear_dual_svc_cv_results} and \ref{nonlinear_lagrangian_dual_svc_cv_results} are obtained with \emph{d} and \emph{r} hyperparameters equal to 3 and 1 respectively for the \emph{polynomial} kernel; \emph{gamma} is setted to \emph{`scale`} for both \emph{polynomial} and \emph{gaussian RBF} kernels. The experiments results shown in \ref{nonlinear_lagrangian_dual_svc_cv_results} are obtained with $\alpha$, i.e., the \emph{learning rate} or \emph{step size}, setted to 0.5 for the \emph{AdaGrad} algorithm.

\begin{table}[H]
\centering
\caption{Nonlinear SVC Wolfe Dual formulation results with Hinge loss}
\label{nonlinear_dual_svc_cv_results}
\begin{tabular}{lllrrrrrr}
\toprule
       &     &     &  fit\_time &  n\_iter &  train\_accuracy &  val\_accuracy &  train\_n\_sv &  val\_n\_sv \\
solver & kernel & C &           &         &                 &               &             &           \\
\midrule
smo & poly & 1   &  0.300656 &      94 &        0.816287 &      0.708488 &          32 &        32 \\
       &     & 10  &  0.304991 &     108 &        0.926131 &      0.743650 &          10 &        10 \\
       &     & 100 &  0.265495 &     172 &        0.957426 &      0.828302 &           8 &         8 \\
       & rbf & 1   &  0.253971 &      50 &        0.998747 &      1.000000 &          43 &        43 \\
       &     & 10  &  0.275428 &      93 &        1.000000 &      1.000000 &          14 &        14 \\
       &     & 100 &  0.220114 &      81 &        1.000000 &      1.000000 &          11 &        11 \\
libsvm & poly & 1   &  0.003757 &     270 &        0.998747 &      0.992481 &          32 &        32 \\
       &     & 10  &  0.003405 &     319 &        1.000000 &      0.992481 &          10 &        10 \\
       &     & 100 &  0.003105 &     274 &        1.000000 &      0.992481 &           8 &         8 \\
       & rbf & 1   &  0.003656 &      99 &        1.000000 &      1.000000 &          44 &        44 \\
       &     & 10  &  0.004566 &     149 &        1.000000 &      1.000000 &          14 &        14 \\
       &     & 100 &  0.003053 &     205 &        1.000000 &      1.000000 &          11 &        11 \\
cvxopt & poly & 1   &  0.085239 &      10 &        0.815039 &      0.705981 &          32 &        32 \\
       &     & 10  &  0.074714 &      10 &        0.926131 &      0.743650 &          10 &        10 \\
       &     & 100 &  0.066410 &      10 &        0.957426 &      0.828302 &           9 &         9 \\
       & rbf & 1   &  0.096586 &      10 &        0.998747 &      1.000000 &          44 &        44 \\
       &     & 10  &  0.082683 &      10 &        1.000000 &      1.000000 &          15 &        15 \\
       &     & 100 &  0.073029 &      10 &        1.000000 &      1.000000 &          14 &        14 \\
\bottomrule
\end{tabular}
\end{table}


\begin{tabular}{lllrrrrr}
\toprule
  ld & kernel &   C &  fit\_time &  train\_accuracy &  val\_accuracy &  nr\_train\_sv &  nr\_val\_sv \\
\midrule
bcqp &   poly &   1 &  0.078685 &        0.750007 &      0.501253 &          217 &        217 \\
  qp &   poly &   1 &  0.737430 &        0.872504 &      0.750627 &          138 &        138 \\
bcqp &    rbf &   1 &  0.028115 &        1.000000 &      0.997512 &          241 &        241 \\
  qp &    rbf &   1 &  1.608392 &        0.800071 &      0.635656 &          188 &        188 \\
bcqp &   poly &  10 &  0.073986 &        0.750007 &      0.501253 &          217 &        217 \\
  qp &   poly &  10 &  0.741292 &        0.872504 &      0.750627 &          138 &        138 \\
bcqp &    rbf &  10 &  0.021263 &        1.000000 &      0.997512 &          241 &        241 \\
  qp &    rbf &  10 &  1.405860 &        0.857500 &      0.718400 &          199 &        199 \\
bcqp &   poly & 100 &  0.063981 &        0.750007 &      0.501253 &          217 &        217 \\
  qp &   poly & 100 &  0.571038 &        0.872504 &      0.750627 &          138 &        138 \\
bcqp &    rbf & 100 &  0.025343 &        1.000000 &      0.997512 &          241 &        241 \\
  qp &    rbf & 100 &  0.761562 &        0.782584 &      0.608218 &          154 &        154 \\
\bottomrule
\end{tabular}


The same considerations made for the previous linear \emph{Wolfe dual} and \emph{Lagrangian dual} formulations are confirmed also in the nonlinearly separable case. In this setting the complexity of the model coming with higher $C$ regularization values seems to be not paying a tradeoff in terms of the number of \emph{iterations} of the algorithm and, moreover, the \emph{bcqp Lagrangian dual} formulation seems to perform better wrt the \emph{qp} formulation, both tends to select even more training data points as \emph{support vectors}.

\subsubsection{Squared Hinge loss}

\paragraph{Primal formulation}

The experiments results shown in \ref{primal_svc_squared_hinge_cv_results} referred to \emph{Stochastic Gradient Descent} algorithm are obtained with $\alpha$, i.e., the \emph{learning rate} or \emph{step size}, setted to 0.001 and $\beta$, i.e., the \emph{momentum}, equal to 0.4. The batch size is setted to 20. Training is stopped if after 5 iterations the training loss is not lower than the best found so far.

\begin{tabular}{lllrrrrrr}
\toprule
   solver & momentum\_type &   C &  fit\_time &  train\_accuracy &  val\_accuracy &  n\_iter &  nr\_train\_sv &  nr\_val\_sv \\
\midrule
liblinear &             - &   1 &  0.004443 &        0.977481 &      0.979949 &     413 &           21 &         10 \\
      sgd &          none &   1 &  0.240039 &        0.970000 &      0.969998 &     130 &           28 &         14 \\
      sgd &      standard &   1 &  0.229616 &        0.970000 &      0.969998 &     102 &           26 &         12 \\
      sgd &      nesterov &   1 &  0.188692 &        0.970000 &      0.969998 &     117 &           24 &         13 \\
liblinear &             - &  10 &  0.006268 &        0.979969 &      0.974898 &    1000 &           17 &          7 \\
      sgd &          none &  10 &  0.039961 &        0.972487 &      0.969998 &      45 &           17 &          9 \\
      sgd &      standard &  10 &  0.022412 &        0.967493 &      0.969998 &      29 &           14 &          7 \\
      sgd &      nesterov &  10 &  0.029363 &        0.969981 &      0.969998 &      38 &           14 &          8 \\
liblinear &             - & 100 &  0.006951 &        0.979969 &      0.969923 &    1000 &           17 &          7 \\
      sgd &          none & 100 &  0.017605 &        0.972487 &      0.969998 &      20 &            9 &          5 \\
      sgd &      standard & 100 &  0.010095 &        0.969981 &      0.964948 &       9 &            3 &          2 \\
      sgd &      nesterov & 100 &  0.018532 &        0.974993 &      0.964948 &      26 &            5 &          2 \\
\bottomrule
\end{tabular}


Again, the results provided from the \emph{custom} implementation, i.e., the SGD with different momentum settings, are strongly similar to those of \emph{sklearn} implementation, i.e., \emph{liblinear} \cite{fan2008liblinear} implementation, in terms of \emph{accuracy} score. More training data points are selected as \emph{support vectors} from the SGD solver but it always requires even lower iterations, i.e., epochs, to achieve the same \emph{numerical precision}. \emph{Standard} or \emph{Polyak} and \emph{Nesterov} momentums always perform lower iterations as expected from the theoretical analysis of the convergence rate.

\pagebreak

\subsection{Support Vector Regression}

Below experiments are about the SVR for which I tested different values for regularization hyperparameter $C$, i.e., from \emph{soft} to \emph{hard margin}, the $\epsilon$ penalty value and in case of nonlinearly separable data also different \emph{kernel functions} mentioned above.

\subsubsection{Epsilon-insensitive loss}

\paragraph{Primal formulation}

The experiments results shown in \ref{primal_svr_eps_cv_results} referred to \emph{Stochastic Gradient Descent} algorithm are obtained with $\alpha$, i.e., the \emph{learning rate} or \emph{step size}, setted to 0.001 and $\beta$, i.e., the \emph{momentum}, equal to 0.4. The batch size is setted to 20. Training is stopped if after 5 iterations the training loss is not lower than the best found so far.

\begin{tabular}{lllrrrrrr}
\toprule
   solver &   C & epsilon &  fit\_time &  train\_r2 &   val\_r2 &  n\_iter &  nr\_train\_sv &  nr\_val\_sv \\
\midrule
  adagrad &   1 &     0.1 &  0.743824 &  0.919139 & 0.915610 &     872 &           66 &         33 \\
liblinear &   1 &     0.1 &  0.000909 &  0.918828 & 0.916845 &      12 &           66 &         33 \\
  adagrad &  10 &     0.1 &  2.723583 &  0.977832 & 0.972861 &    3546 &           65 &         32 \\
liblinear &  10 &     0.1 &  0.001047 &  0.977848 & 0.972083 &     122 &           66 &         33 \\
  adagrad & 100 &     0.1 &  3.106563 &  0.978115 & 0.974270 &    3999 &           66 &         32 \\
liblinear & 100 &     0.1 &  0.001624 &  0.977723 & 0.974270 &     735 &           65 &         33 \\
  adagrad &   1 &     0.2 &  0.738437 &  0.919988 & 0.916497 &     885 &           66 &         33 \\
liblinear &   1 &     0.2 &  0.000924 &  0.918753 & 0.916601 &      13 &           65 &         32 \\
  adagrad &  10 &     0.2 &  2.604949 &  0.977798 & 0.972835 &    3514 &           65 &         32 \\
liblinear &  10 &     0.2 &  0.001276 &  0.977852 & 0.972041 &     164 &           64 &         33 \\
  adagrad & 100 &     0.2 &  3.065408 &  0.978128 & 0.974186 &    3999 &           66 &         32 \\
liblinear & 100 &     0.2 &  0.001437 &  0.977641 & 0.973867 &     693 &           66 &         33 \\
  adagrad &   1 &     0.3 &  0.698630 &  0.920124 & 0.916702 &     878 &           65 &         33 \\
liblinear &   1 &     0.3 &  0.001215 &  0.919287 & 0.917030 &      10 &           66 &         32 \\
  adagrad &  10 &     0.3 &  2.715705 &  0.977781 & 0.972876 &    3458 &           65 &         32 \\
liblinear &  10 &     0.3 &  0.001184 &  0.977871 & 0.972149 &     129 &           63 &         33 \\
  adagrad & 100 &     0.3 &  2.610475 &  0.978122 & 0.974216 &    3999 &           66 &         32 \\
liblinear & 100 &     0.3 &  0.001536 &  0.977641 & 0.973903 &     808 &           65 &         33 \\
\bottomrule
\end{tabular}


\paragraph{Linear Dual formulations}

The experiments results shown in \ref{linear_lagrangian_dual_svr_cv_results} are obtained with $\alpha$, i.e., the \emph{learning rate} or \emph{step size}, setted to 0.5 for the \emph{AdaGrad} algorithm. Notice that the \emph{qp} dual refers to the formulation \eqref{eq:svr_lagrangian_dual}, while the \emph{bcqp} dual refers to the formulation \eqref{eq:svr_bcqp_lagrangian_dual}.

\begin{table}[H]
\centering
\caption{Linear SVR Wolfe Dual formulation results with Epsilon-insensitive loss}
\label{linear_dual_svr_cv_results}
\begin{tabular}{lllrrrrrr}
\toprule
       &     &     &  fit\_time &  n\_iter &  train\_r2 &    val\_r2 &  train\_n\_sv &  val\_n\_sv \\
solver & C & epsilon &           &         &           &           &             &           \\
\midrule
smo & 10  & 0.2 &  0.041816 &      30 &  0.839199 &  0.824149 &          67 &        67 \\
       & 100 & 0.3 &  0.231624 &     136 &  0.838683 &  0.821843 &          66 &        66 \\
       &     & 0.1 &  0.161679 &     128 &  0.838229 &  0.820990 &          66 &        66 \\
       & 10  & 0.3 &  0.037710 &      35 &  0.839234 &  0.824092 &          67 &        67 \\
       &     & 0.1 &  0.039658 &      30 &  0.839160 &  0.824200 &          67 &        67 \\
       & 1   & 0.3 &  0.015873 &      12 &  0.816829 &  0.806018 &          66 &        66 \\
       &     & 0.2 &  0.016650 &      13 &  0.816266 &  0.805431 &          66 &        66 \\
       &     & 0.1 &  0.018286 &      13 &  0.815865 &  0.804998 &          66 &        66 \\
       & 100 & 0.2 &  0.206949 &     185 &  0.838469 &  0.821431 &          66 &        66 \\
libsvm & 1   & 0.3 &  0.003237 &      57 &  0.816092 &  0.803425 &          66 &        66 \\
       & 10  & 0.1 &  0.003931 &     111 &  0.837101 &  0.825608 &          67 &        67 \\
       &     & 0.2 &  0.001921 &     114 &  0.837315 &  0.825348 &          67 &        67 \\
       &     & 0.3 &  0.001678 &     163 &  0.837508 &  0.825064 &          67 &        67 \\
       & 1   & 0.1 &  0.003243 &      52 &  0.815113 &  0.803346 &          66 &        66 \\
       & 100 & 0.1 &  0.002190 &    1081 &  0.836676 &  0.824330 &          66 &        66 \\
       &     & 0.2 &  0.001872 &     762 &  0.836958 &  0.824744 &          66 &        66 \\
       & 1   & 0.2 &  0.005966 &      52 &  0.815554 &  0.803517 &          66 &        66 \\
       & 100 & 0.3 &  0.001949 &    1433 &  0.837205 &  0.825121 &          66 &        66 \\
cvxopt & 1   & 0.2 &  0.018971 &      10 &  0.816274 &  0.805607 &          66 &        66 \\
       & 100 & 0.3 &  0.010127 &       8 &  0.838686 &  0.822088 &          67 &        67 \\
       &     & 0.2 &  0.014582 &       8 &  0.838473 &  0.821675 &          67 &        67 \\
       &     & 0.1 &  0.012519 &       8 &  0.838233 &  0.821232 &          67 &        67 \\
       & 10  & 0.3 &  0.013504 &       8 &  0.839237 &  0.824234 &          67 &        67 \\
       &     & 0.2 &  0.013448 &       8 &  0.839203 &  0.824285 &          67 &        67 \\
       &     & 0.1 &  0.016795 &       8 &  0.839164 &  0.824329 &          67 &        67 \\
       & 1   & 0.3 &  0.012157 &       9 &  0.816837 &  0.806148 &          67 &        67 \\
       &     & 0.1 &  0.021911 &       9 &  0.815872 &  0.805162 &          67 &        67 \\
\bottomrule
\end{tabular}
\end{table}


\begin{table}[h!]
\centering
\caption{Linear SVR Lagrangian Dual formulation results with Epsilon-insensitive loss}
\label{linear_lagrangian_dual_svr_cv_results}
\begin{tabular}{lllrrrrrr}
\toprule
   &     &     &  fit\_time &  n\_iter &  train\_r2 &    val\_r2 &  train\_n\_sv &  val\_n\_sv \\
dual & C & epsilon &           &         &           &           &             &           \\
\midrule
bcqp & 1   & 0.1 &  0.656280 &     522 &  0.731073 &  0.721200 &          67 &        67 \\
   &     & 0.2 &  0.658018 &     524 &  0.731073 &  0.721199 &          67 &        67 \\
   &     & 0.3 &  0.602584 &     526 &  0.731073 &  0.721199 &          67 &        67 \\
   & 10  & 0.1 &  0.614027 &     539 &  0.733638 &  0.723925 &          67 &        67 \\
   &     & 0.2 &  0.668057 &     541 &  0.733638 &  0.723924 &          67 &        67 \\
   &     & 0.3 &  0.637837 &     543 &  0.733638 &  0.723924 &          67 &        67 \\
   & 100 & 0.1 &  0.733603 &     539 &  0.733638 &  0.723925 &          67 &        67 \\
   &     & 0.2 &  0.627369 &     541 &  0.733638 &  0.723924 &          67 &        67 \\
   &     & 0.3 &  0.409460 &     543 &  0.733638 &  0.723924 &          67 &        67 \\
qp & 1   & 0.1 &  0.674449 &     653 &  0.876534 &  0.870926 &          67 &        67 \\
   &     & 0.2 &  0.678396 &     653 &  0.876534 &  0.870927 &          67 &        67 \\
   &     & 0.3 &  0.722974 &     653 &  0.876534 &  0.870927 &          67 &        67 \\
   & 10  & 0.1 &  0.498502 &     519 &  0.731825 &  0.722021 &          67 &        67 \\
   &     & 0.2 &  0.563454 &     524 &  0.731825 &  0.722021 &          67 &        67 \\
   &     & 0.3 &  0.537692 &     530 &  0.731825 &  0.722020 &          67 &        67 \\
   & 100 & 0.1 &  0.652838 &     519 &  0.731825 &  0.722021 &          67 &        67 \\
   &     & 0.2 &  0.604621 &     524 &  0.731825 &  0.722021 &          67 &        67 \\
   &     & 0.3 &  0.550931 &     530 &  0.731825 &  0.722020 &          67 &        67 \\
\bottomrule
\end{tabular}
\end{table}


\paragraph{Nonlinear Dual formulations}

The experiments results shown in \ref{nonlinear_dual_svr_cv_results} and \ref{nonlinear_lagrangian_dual_svr_cv_results} are obtained with \emph{d} and \emph{r} hyperparameters both equal to 3 for the \emph{polynomial} kernel; \emph{gamma} is setted to \emph{`scale`} for both \emph{polynomial} and \emph{gaussian RBF} kernels. The experiments results shown in \ref{nonlinear_lagrangian_dual_svc_cv_results} are obtained with $\alpha$, i.e., the \emph{learning rate} or \emph{step size}, setted to 0.5 for the \emph{AdaGrad} algorithm.

\begin{table}[H]
\centering
\caption{Nonlinear SVR Wolfe Dual formulation results with Epsilon-insensitive loss}
\label{nonlinear_dual_svr_cv_results}
\begin{tabular}{llllrrrr}
\toprule
       &     &     &     &     fit\_time &        r2 &    n\_iter &  n\_sv \\
solver & kernel & C & epsilon &              &           &           &       \\
\midrule
smo & poly & 1   & 0.1 &    16.621604 &  0.965958 &     22565 &    28 \\
       &     &     & 0.2 &    10.662649 &  0.915386 &     18370 &     7 \\
       &     &     & 0.3 &     2.206419 & -0.019348 &      2577 &     4 \\
       &     & 10  & 0.1 &   162.003244 &  0.873457 &    269916 &    29 \\
       &     &     & 0.2 &     5.475889 &  0.773653 &      7385 &     4 \\
       &     &     & 0.3 &     2.235514 & -0.019348 &      2577 &     4 \\
       &     & 100 & 0.1 &  1570.878271 &  0.873592 &   2837623 &    29 \\
       &     &     & 0.2 &     5.905676 &  0.773653 &      7385 &     4 \\
       &     &     & 0.3 &     1.954492 & -0.019348 &      2577 &     4 \\
       & rbf & 1   & 0.1 &     0.162744 &  0.989251 &        58 &    20 \\
       &     &     & 0.2 &     0.042457 &  0.916460 &        30 &     6 \\
       &     &     & 0.3 &     0.017082 &  0.863690 &         7 &     5 \\
       &     & 10  & 0.1 &     0.962114 &  0.988808 &       855 &    22 \\
       &     &     & 0.2 &     0.023530 &  0.916260 &        26 &     6 \\
       &     &     & 0.3 &     0.010618 &  0.863690 &         7 &     5 \\
       &     & 100 & 0.1 &     4.986097 &  0.986641 &      4714 &    19 \\
       &     &     & 0.2 &     0.019692 &  0.916260 &        26 &     6 \\
       &     &     & 0.3 &     0.017319 &  0.863690 &         7 &     5 \\
libsvm & poly & 1   & 0.1 &     0.033172 &  0.982113 &     58829 &    28 \\
       &     &     & 0.2 &     0.010934 &  0.974153 &     15817 &     7 \\
       &     &     & 0.3 &     0.009176 &  0.946442 &      2697 &     4 \\
       &     & 10  & 0.1 &     0.166147 &  0.982366 &    477441 &    29 \\
       &     &     & 0.2 &     0.008310 &  0.979179 &      5598 &     4 \\
       &     &     & 0.3 &     0.005829 &  0.946442 &      2697 &     4 \\
       &     & 100 & 0.1 &     2.230161 &  0.982119 &  10669851 &    29 \\
       &     &     & 0.2 &     0.020086 &  0.979179 &      5598 &     4 \\
       &     &     & 0.3 &     0.004748 &  0.946442 &      2697 &     4 \\
       & rbf & 1   & 0.1 &     0.009822 &  0.989178 &        80 &    20 \\
       &     &     & 0.2 &     0.013293 &  0.982040 &        37 &     6 \\
       &     &     & 0.3 &     0.001457 &  0.951730 &        18 &     5 \\
       &     & 10  & 0.1 &     0.003906 &  0.990056 &       833 &    20 \\
       &     &     & 0.2 &     0.007880 &  0.982014 &        38 &     6 \\
       &     &     & 0.3 &     0.002544 &  0.951730 &        18 &     5 \\
       &     & 100 & 0.1 &     0.011858 &  0.990542 &     12361 &    19 \\
       &     &     & 0.2 &     0.005540 &  0.982014 &        38 &     6 \\
       &     &     & 0.3 &     0.001965 &  0.951730 &        18 &     5 \\
cvxopt & poly & 1   & 0.1 &     0.087879 &  0.966650 &         8 &    28 \\
       &     &     & 0.2 &     0.162963 &  0.389408 &        10 &     6 \\
       &     &     & 0.3 &     0.103764 &  0.070818 &        10 &     4 \\
       &     & 10  & 0.1 &     0.133711 &  0.824379 &         9 &    30 \\
       &     &     & 0.2 &     0.139223 &  0.777645 &        10 &     4 \\
       &     &     & 0.3 &     0.069593 &  0.070820 &        10 &     4 \\
       &     & 100 & 0.1 &     0.044351 &  0.952412 &         9 &    72 \\
       &     &     & 0.2 &     0.033852 &  0.777643 &        10 &     4 \\
       &     &     & 0.3 &     0.036072 &  0.070818 &        10 &     4 \\
       & rbf & 1   & 0.1 &     0.042391 &  0.989236 &        10 &    20 \\
       &     &     & 0.2 &     0.033710 &  0.915828 &         9 &     6 \\
       &     &     & 0.3 &     0.050085 &  0.858355 &         9 &     6 \\
       &     & 10  & 0.1 &     0.043324 &  0.987421 &        10 &    21 \\
       &     &     & 0.2 &     0.032647 &  0.915828 &        10 &     6 \\
       &     &     & 0.3 &     0.044080 &  0.861873 &        10 &     5 \\
       &     & 100 & 0.1 &     0.043129 &  0.990163 &        10 &    29 \\
       &     &     & 0.2 &     0.025641 &  0.915828 &        10 &     6 \\
       &     &     & 0.3 &     0.046823 &  0.861873 &        10 &     5 \\
\bottomrule
\end{tabular}
\end{table}


\begin{tabular}{llllrrrrr}
\toprule
  ld & kernel &   C & epsilon &  fit\_time &  train\_r2 &     val\_r2 &  nr\_train\_sv &  nr\_val\_sv \\
\midrule
bcqp &   poly &   1 &     0.1 &  0.023057 &  0.639114 & -36.677747 &           67 &         67 \\
  qp &   poly &   1 &     0.1 &  0.020220 &  0.646376 & -11.830936 &           67 &         67 \\
bcqp &    rbf &   1 &     0.1 &  0.064591 &  0.733892 &  -3.633330 &           67 &         67 \\
  qp &    rbf &   1 &     0.1 &  0.262468 &  0.705219 &  -4.814520 &           67 &         67 \\
bcqp &   poly &  10 &     0.1 &  0.023814 &  0.639114 & -36.677747 &           67 &         67 \\
  qp &   poly &  10 &     0.1 &  0.022628 &  0.646376 & -11.830936 &           67 &         67 \\
bcqp &    rbf &  10 &     0.1 &  0.060908 &  0.733892 &  -3.633330 &           67 &         67 \\
  qp &    rbf &  10 &     0.1 &  0.111514 &  0.683448 &  -5.253019 &           67 &         67 \\
bcqp &   poly & 100 &     0.1 &  0.021451 &  0.639114 & -36.677747 &           67 &         67 \\
  qp &   poly & 100 &     0.1 &  0.019278 &  0.646376 & -11.830936 &           67 &         67 \\
bcqp &    rbf & 100 &     0.1 &  0.049753 &  0.733892 &  -3.633330 &           67 &         67 \\
  qp &    rbf & 100 &     0.1 &  0.113938 &  0.683448 &  -5.253019 &           67 &         67 \\
bcqp &   poly &   1 &     0.2 &  0.028676 &  0.617963 & -26.867830 &           66 &         66 \\
  qp &   poly &   1 &     0.2 &  0.060558 &  0.646709 & -11.845840 &           67 &         67 \\
bcqp &    rbf &   1 &     0.2 &  0.158712 &  0.644671 &  -4.372948 &           67 &         67 \\
  qp &    rbf &   1 &     0.2 &  0.317687 &  0.697766 &  -4.973421 &           67 &         67 \\
bcqp &   poly &  10 &     0.2 &  0.023270 &  0.617963 & -26.867830 &           66 &         66 \\
  qp &   poly &  10 &     0.2 &  0.048967 &  0.646709 & -11.845840 &           67 &         67 \\
bcqp &    rbf &  10 &     0.2 &  0.155017 &  0.644671 &  -4.372948 &           67 &         67 \\
  qp &    rbf &  10 &     0.2 &  0.167689 &  0.664793 &  -5.391712 &           67 &         67 \\
bcqp &   poly & 100 &     0.2 &  0.022938 &  0.617963 & -26.867830 &           66 &         66 \\
  qp &   poly & 100 &     0.2 &  0.052707 &  0.646709 & -11.845840 &           67 &         67 \\
bcqp &    rbf & 100 &     0.2 &  0.127478 &  0.644671 &  -4.372948 &           67 &         67 \\
  qp &    rbf & 100 &     0.2 &  0.154184 &  0.664793 &  -5.391712 &           67 &         67 \\
bcqp &   poly &   1 &     0.3 &  0.060247 &  0.591564 & -26.749052 &           66 &         66 \\
  qp &   poly &   1 &     0.3 &  0.059887 &  0.623022 & -11.794656 &           67 &         67 \\
bcqp &    rbf &   1 &     0.3 &  0.252759 &  0.549688 &  -5.236443 &           67 &         67 \\
  qp &    rbf &   1 &     0.3 &  0.456361 &  0.683198 &  -5.011083 &           67 &         67 \\
bcqp &   poly &  10 &     0.3 &  0.049147 &  0.591564 & -26.749052 &           66 &         66 \\
  qp &   poly &  10 &     0.3 &  0.056914 &  0.623022 & -11.794656 &           67 &         67 \\
bcqp &    rbf &  10 &     0.3 &  0.231289 &  0.549688 &  -5.236443 &           67 &         67 \\
  qp &    rbf &  10 &     0.3 &  0.221825 &  0.672122 &  -5.178610 &           67 &         67 \\
bcqp &   poly & 100 &     0.3 &  0.043412 &  0.591564 & -26.749052 &           66 &         66 \\
  qp &   poly & 100 &     0.3 &  0.079256 &  0.623022 & -11.794656 &           67 &         67 \\
bcqp &    rbf & 100 &     0.3 &  0.180342 &  0.549688 &  -5.236443 &           67 &         67 \\
  qp &    rbf & 100 &     0.3 &  0.157599 &  0.672122 &  -5.178610 &           67 &         67 \\
\bottomrule
\end{tabular}


\newpage

\subsubsection{Squared Epsilon-insensitive loss}

\paragraph{Primal formulation}

The experiments results shown in \ref{primal_svr_squared_eps_cv_results} referred to \emph{Stochastic Gradient Descent} algorithm are obtained with $\alpha$, i.e., the \emph{learning rate} or \emph{step size}, setted to 0.001 and $\beta$, i.e., the \emph{momentum}, equal to 0.4. The batch size is setted to 20. Training is stopped if after 5 iterations the training loss is not lower than the best found so far.

\begin{table}[h!]
\centering
\caption{SVR Primal formulation results with Squared Epsilon-insensitive loss}
\label{primal_svr_squared_eps_cv_results}
\begin{tabular}{llllrrrrrr}
\toprule
          &     &   &     &  fit\_time &  n\_iter &     train\_r2 &       val\_r2 &  train\_n\_sv &  val\_n\_sv \\
solver & C & momentum & epsilon &           &         &              &              &             &           \\
\midrule
gd & 1   & nesterov & 0.1 &  0.120714 &     183 &     0.978130 &     0.973982 &          66 &        32 \\
          &     &   & 0.2 &  0.098704 &     181 &     0.978129 &     0.973979 &          66 &        32 \\
          &     &   & 0.3 &  0.108854 &     179 &     0.978129 &     0.973978 &          66 &        32 \\
          &     & none & 0.1 &  0.217456 &     352 &     0.978126 &     0.973976 &          66 &        32 \\
          &     &   & 0.2 &  0.200618 &     349 &     0.978125 &     0.973974 &          66 &        32 \\
          &     &   & 0.3 &  0.172997 &     345 &     0.978124 &     0.973972 &          66 &        32 \\
          &     & standard & 0.1 &  0.126630 &     180 &     0.978130 &     0.973982 &          66 &        32 \\
          &     &   & 0.2 &  0.112161 &     177 &     0.978129 &     0.973979 &          66 &        32 \\
          &     &   & 0.3 &  0.094951 &     176 &     0.978129 &     0.973978 &          66 &        32 \\
          & 10  & nesterov & 0.1 &  0.014011 &      26 &     0.978184 &     0.973958 &          66 &        33 \\
          &     &   & 0.2 &  0.013980 &      25 &     0.978184 &     0.973958 &          66 &        33 \\
          &     &   & 0.3 &  0.011135 &      25 &     0.978184 &     0.973958 &          66 &        33 \\
          &     & none & 0.1 &  0.025574 &      48 &     0.978184 &     0.973958 &          66 &        33 \\
          &     &   & 0.2 &  0.024828 &      46 &     0.978184 &     0.973957 &          66 &        33 \\
          &     &   & 0.3 &  0.024074 &      45 &     0.978183 &     0.973955 &          66 &        33 \\
          &     & standard & 0.1 &  0.013464 &      25 &     0.977876 &     0.975097 &          65 &        33 \\
          &     &   & 0.2 &  0.013465 &      25 &     0.977868 &     0.975106 &          65 &        33 \\
          &     &   & 0.3 &  0.013375 &      25 &     0.977876 &     0.975092 &          65 &        33 \\
          & 100 & nesterov & 0.1 &  0.003692 &       6 & -1638.307583 & -1608.863197 &          67 &        33 \\
          &     &   & 0.2 &  0.003719 &       6 & -1622.486618 & -1592.867828 &          67 &        33 \\
          &     &   & 0.3 &  0.002775 &       6 & -1651.292447 & -1620.647103 &          67 &        33 \\
          &     & none & 0.1 &  0.003545 &       6 &   -19.106771 &   -18.943994 &          67 &        33 \\
          &     &   & 0.2 &  0.003577 &       6 &   -18.905404 &   -18.746050 &          67 &        33 \\
          &     &   & 0.3 &  0.003799 &       6 &   -18.834177 &   -18.679996 &          67 &        33 \\
          &     & standard & 0.1 &  0.018089 &      29 &     0.978184 &     0.973963 &          66 &        33 \\
          &     &   & 0.2 &  0.016128 &      29 &     0.978184 &     0.973963 &          66 &        33 \\
          &     &   & 0.3 &  0.013552 &      29 &     0.978184 &     0.973968 &          66 &        33 \\
liblinear & 1   & - & 0.1 &  0.000897 &      86 &     0.978134 &     0.974000 &          67 &        32 \\
          &     &   & 0.2 &  0.000700 &      83 &     0.978132 &     0.974006 &          66 &        32 \\
          &     &   & 0.3 &  0.000768 &      89 &     0.978130 &     0.974014 &          66 &        32 \\
          & 10  & - & 0.1 &  0.002579 &     768 &     0.978183 &     0.973965 &          66 &        33 \\
          &     &   & 0.2 &  0.002345 &     757 &     0.978183 &     0.973979 &          66 &        33 \\
          &     &   & 0.3 &  0.002486 &     752 &     0.978183 &     0.973977 &          66 &        32 \\
          & 100 & - & 0.1 &  0.003818 &    1000 &     0.978079 &     0.972999 &          66 &        33 \\
          &     &   & 0.2 &  0.003834 &    1000 &     0.977766 &     0.971757 &          66 &        32 \\
          &     &   & 0.3 &  0.003442 &    1000 &     0.978113 &     0.974832 &          66 &        32 \\
\bottomrule
\end{tabular}
\end{table}
